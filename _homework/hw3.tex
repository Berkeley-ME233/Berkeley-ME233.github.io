\documentclass[letterpaper,12pt]{article}
\usepackage{amsmath}
\usepackage{amssymb}
\usepackage{amsthm}
\usepackage[dvips]{graphicx}
%\usepackage{subfig}
\usepackage[margin=2.5cm,nohead]{geometry}
\usepackage{url}
\usepackage{verbatim}

%\input{../commands}
%\setkeys{Gin}{draft=true}
%
%\newtheorem{thm}{Theorem}%[subsection]
%\newtheorem{lem}[thm]{Lemma}
%\newtheorem{cor}[thm]{Corollary}
%\newtheorem{prop}[thm]{Proposition}
%\newtheorem{remark}[thm]{Remark}
%\newtheorem*{thm*}{Theorem}
%\newtheorem*{lem*}{Lemma}



\begin{document}

\begin{center}
    {\bf UNIVERSITY OF CALIFORNIA AT BERKELEY}\\
    {\bf Department of Mechanical Engineering}\\
    {\bf ME233  Advanced Control Systems II}\\
    Spring 2016\\
\end{center}
\noindent
{\Large \bf Homework \#3 }\\[-3em]
\begin{flushright}
\begin{tabular} {l l}
    Assigned: &  Mar.\ 2 (Wed)\\
    Due: & Mar.\ 8 (Tu)
\end{tabular}
\end{flushright}

\begin{enumerate}



\item
In this problem we will verify some results concerning the convergence of the LQR's discrete Riccati equation (DRE) to a steady state solution and the existence, uniqueness and closed loop stability of the discrete algebraic Riccati equation (DARE) solution.


Consider the design of an optimal LQR for the LTI discrete-time system
\begin{align}
    \label{abc1}
    x(k+1) & = A \: x(k) + B \: u(k) \\
    y(k) & = C \: x(k) \, \nonumber
\end{align}
where $u(k) = -K(k+1) \: x(k)$ is the optimal control input that minimizes the following cost criteria
\begin{align*}
    %\label{cost1}
    J[x_o, m, Q_{_f},N] = x^T(N)\, Q_{_f} \, x(N) + \sum_{k=m}^{N-1} \: \left \{ y^2(k) + R\,u^2(k) \right \}
        \quad \textrm{s.t.} \quad x(m) = x_o\,,
\end{align*}
for $m = 0$, $Q_{_f} = Q_{_f}^T \succeq 0$ and $R = R^T \succ 0$, and any arbitrary initial condition $x_o \in \mathcal{R}^n$.

Define the optimal value function
\begin{align*}
    J^o[x_o, m, Q_{_f},N] = \min_{U_{[m,N-1]}} \, J[x_o, m, Q_{_f},N]
\end{align*}
where $U_{[m,N-1]} = \left \{ u(m),\, \cdots , \, u(N-1)\right \}$ is the set of all possible control actions from $k=m$ to $k=N-1$.

\begin{enumerate}

\item
\label{case:a}
Let
\begin{align*}
    A = \begin{bmatrix}
            1.2  & 0 & 0 \\
            0 & 0 & 1 \\
            0 & -2 & 0
        \end{bmatrix} \hspace{2em}
        B = \begin{bmatrix}
            1 & 0 \\
            0 & 0 \\
            0 & 1 \end{bmatrix} \hspace{2em}
        C = \begin{bmatrix}
            10 & 0 & 0
        \end{bmatrix} \hspace{2em} R = \begin{bmatrix}
            10 & 0 \\
            0 & 10
        \end{bmatrix}
\end{align*}
and verify that $(A,B)$ is stabilizable but $(C,A)$ is not detectable. Let $P(N) = Q_{_f}$. For each of the four cases
\begin{enumerate}
    \item $Q_{_f} = 0$
    \item $Q_{_f} = {\rm diag}(0,\,0,\,1)$,
    \item $Q_{_f} = {\rm diag}(1,\,1,\,1)$,
    \item $Q_{_f} = {\rm diag}(10,\,1,\,1)$,
\end{enumerate}
do the following:

\begin{itemize}
    \item
    For $x_o = [1 \ \ 0 \ \ 1]^T$, plot $J^o[x_o, N - m, Q_{_f}, N] $  vs $m$. (Note that this will require computing $P(k)$, the solution of the Riccati difference equation, backwards from $P(50) = Q_{_f}$.)

    \item
    Compute the solution of the DARE %using the MATLAB command {\tt dare}
symbolically in terms of unknown elements of $P_\infty$
and compare it with values of $P(0)$ and $P(1)$.

\end{itemize}
Discuss your results.


%\item
%Let
%\begin{align*}
%    A = \begin{bmatrix}
%            1.2  & 1 & 0 \\
%            0 & 0 & 1 \\
%            0 & -2 & 0
%        \end{bmatrix}
%\end{align*}
%and everything else be the same as in part (a). Repeat part (a) %for the two cases
%\begin{enumerate}
%    \item $Q_{_f} = 0$
%    \item $Q_{_f} = {\rm diag}(0,\,0,\,1)$.
%\end{enumerate}
%Discuss your results.

\item
Let
\begin{align*}
    A & = \begin{bmatrix}
            0.8 & 1 & 1 \\
            0 & -1 & 1 \\
            0 & 0 & -1
        \end{bmatrix}
        & B & = \begin{bmatrix}
            0 \\
            0 \\
            1
        \end{bmatrix}
        & C & = \begin{bmatrix}
            0 & 1 & 0
        \end{bmatrix}
        & R & = 0.1 \; .
\end{align*}
Repeat part (a) for the two cases

\begin{enumerate}
    \item $Q_{_f} = 0$
    \item $Q_{_f} = {\rm diag}(0,\,0,\,1)$.
\end{enumerate}
Discuss your results.

\end{enumerate}





\item

Consider the system
\begin{align*}
    \begin{bmatrix}
            x_1(k+1) \\
            x_2(k+1)
        \end{bmatrix} & = \begin{bmatrix}
            -0.08 & -1 \\
            0.7 & 0.1
        \end{bmatrix} \begin{bmatrix}
            x_1(k) \\
            x_2(k)
        \end{bmatrix} + \begin{bmatrix}
            0.34 \\
            0.3
        \end{bmatrix} (u(k) + w(k)) \\
    y(k) & = \begin{bmatrix}
            0 & 3
        \end{bmatrix} \begin{bmatrix}
            x_1(k) \\
            x_2(k)
        \end{bmatrix} + v(k)
\end{align*}
where $u(k)$ is a deterministic input and
\begin{itemize}
    \item
    $E \{ X(0) \} = \begin{bmatrix} 0 \\ 0 \end{bmatrix}$ and $E \{ X(0) X^T(0) \} = \begin{bmatrix} 0.1 & 0 \\ 0 & 0.1 \end{bmatrix}$

    \item
    $X(0)$ is Gaussian and $W(k)$ and $V(k)$ are white Gaussian sequences

    \item
    $m_{w} = E \{ W(k) \} = 10, \quad E \{ V(k) \} = 0$

    \item
    $E \left\{ \begin{bmatrix}
        W(k+j) - m_w  \\
        V(k+j)
    \end{bmatrix} \begin{bmatrix}
        (W(k) - m_w) &
        V(k)
    \end{bmatrix} \right\} = \begin{bmatrix}
        W & 0 \\
        0 & V
    \end{bmatrix} \delta(j) = \begin{bmatrix}
        1 & 0 \\
        0 & 0.5
    \end{bmatrix} \delta(j)$

    \item
    $E \left\{ \begin{bmatrix}
        W(k) - m_w  \\
        V(k)
    \end{bmatrix} X^T(0) \right\} = 0$

\end{itemize}
A Kalman filter is used to estimate the state of the system using the measurement sequence $y(k)$.

\begin{enumerate}

    \item
    Find the steady state values of the following time varying matrices and scalars:
    \begin{itemize}

        \item
        The a priori state estimation error covariance $M(k)$

        \item
        The a posteriori state estimation error covariance $Z(k)$

        \item
        The a priori output estimation error covariance
        \begin{align*}
            \Lambda_{\tilde{y}^o \tilde{y}^o}(k,0) = E \{ | \tilde{y}^o(k) |^2 \} = CM(k) C^T + V \; .
        \end{align*}

        \item
        The Kalman filter gains $L(k)$ and $F(k)$

    \end{itemize}
    You should find the steady state values of these quantities by recursively computing their values forwards in time until they converge to their respective steady state values.

    \item
    Plot the response of $\Lambda_{\tilde{y}^o \tilde{y}^o}(k,0)$.
\end{enumerate}
%\balphlist
%\item
%Starting from the initial conditions, recursively compute the following time varying matrices and scalars, until their respective steady state values are reached:
%\bitem
%\item
%The a-priori and a-posteriori state estimation error covariances, respectively $M(k)$ and $Z(k)$.
%\item
%The  a-priori and a-posteriorly output estimation error covariances, respectively
%\footnote{The a-priori output estimation error $\yt^o(k) = y(k) - \yh^o(k)$ is often called the {\em innovation sequence}.}
%\beqns
%\Lambda_{\yt^o \yt^o }(k) &=& E \{ |\yt^o(k)|^2\} = C M(k) C^T + V,\:\:\:
%\yt^o(k) = y(k) - \yh^o(k)\\
%\Lambda_{\yt  \yt  }(k) &=& E \{ |\yt (k)|^2\} = C Z(k) C^T - V
%+ 2V[CM(k) C +V]^{-1} V,\:\:\:\\\yt(k) &=& y(k) - \yh(k)\\
%\eeqns
%%\item
%%The Kalman filter feedback gains $L(k) =  A M(k) C^T   [ C M(k) C^T + R ] ^{-1}$  and $F(k) = M(k) C^T   [ C M(k) C^T + R ] ^{-1}$
%%\footnote{$F(k)$ is often called the innovation gain.}.
%%\noindent
%\eitem
%
%\item Plot the response of $\Lambda_{\yt^o \yt^o }(k)$.


%\item
%Compute the steady state covariances $M$, $Z$, $\Lambda_{\yt^o \yt^o }$ and $\Lambda_{\yt  \yt  }$, and Kalman filter gains $L$ and $F$ by solving the algebraic Riccati equation
%\beqns
%A M A^T  - M +   B W B^{T}   - A  M  C^T
%    \left [  C M  C^T + V \right ]^{-1}  C M  A^T  = 0
%\eeqns
%using either the Matlab commands {\tt dare} or {\tt kalman}. Read the Matlab help entries on these functions to understand what they do. Also compute the close loop eigenvalues $A_c = A - L C$.
%\item
%Perform a simulation by running the plant and the {\bf steady state} Kalman filter (i.e. only use the steady state Kalman filter gains $L$ and $F$)
%\footnote{Notice that the steady state Kalman filter is returned as the first output argument of the matlab function {\tt kalman}.} for a sufficiently long period of time, so that expected values and covariances can be approximated by time averages. Use $u(k) = 10$ as the deterministic input and generate a set of sample outcomes $x(0)$, $w(k)$ and $v(k)$.
%\item
%From the resulting simulation data, calculate, using the function  {\tt cov}, the approximate steady state values of $\Lyy$, $\Lambda_{\yt^o  \yt^o  }$ and $\Lambda_{\yt  \yt  }$, and compare them with their actual values, which were previously calculated.
%\item Repeat steps (a)-(e) but change the measurement noise intensity as follows i) $V = 0.05$, ii) $V = 5$. Discuss your results for all three cases.





\item
{\bf Kalman filter with correlated input and measurement noise:}

Consider the discrete-time system given by
\begin{align}
    x(k+1) & = A x(k) + B u(k) + w(k)
        \label{ds1} \\
    y(k) & = C x(k) + v(k)
        \label{ds2}
\end{align}
where $E \{x(0)\} = x_o$, $E\{ w(k) \} = 0$, $E\{v(k)\} = 0$, $E \{ (x(0) - x_o)(x(0) - x_o)^T \} = X_o$, $E\{ (x(0) - x_o) w^T(k) \} = 0$, $E \{(x(0) - x_o) v^T(k) \} = 0$, and
\begin{align*}
    E \left \{ \begin{bmatrix}
            w(k) \\
            v(k)
        \end{bmatrix} \begin{bmatrix}
            w(j)^T & v(j)^T
        \end{bmatrix} \right \} = \begin{bmatrix}
            W & S \\
            S^T & V
        \end{bmatrix} \, \delta(k-j)
\end{align*}
%and $W \in \mathbb{R} ^{n \times n}$ is a positive semi-definite matrix
where $V \in \mathbb{R} ^{m \times m}$ is positive definite matrix. The a-priori Kalman filter for this system can be written as
\begin{align}
    \hat{x}^o(k+1) & = A \hat{x}^o(k) + B u(k) + L(k) [ y(k) - C \hat{x}^o(k) ]
        \label{kf-1} \\
    L(k) & = [ A M(k) C^T + S ] [ C M(k) C^T + V ] ^{-1}\\
    M(k+1) & = A M(k) A^T + W - [AM(k) C^T + S] [ C M(k) C^T + V ]^{-1} [CM(k) A^T + S^T]
        \label{kf-3}
\end{align}
with initial conditions $\hat{x}^o(0) = x_o$ and $M(0) = X_o$.

Derive Eqs.~\eqref{kf-1}--\eqref{kf-3} using previously-derived results in Kalman filtering and noticing that Eqs.~\eqref{ds1}--\eqref{ds2} can be written as
\begin{align*}
    x(k+1) & = A^{'} x(k) + B u(k) + w^{'}(k) + SV^{-1} y(k) \, ,
\end{align*}
where $A^{'} = A - SV^{-1}C$ and
\begin{align*}
    E \left \{ \begin{bmatrix}
            w^{'}(k) \\
            v(k)
        \end{bmatrix} \begin{bmatrix}
            w^{'}(j)^T & v(j)^T
        \end{bmatrix} \right \} = \begin{bmatrix}
            W^{'} & 0 \\
            0 & V
        \end{bmatrix} \, \delta(k-j),\hspace{3em}W^{'} = W - S V^{-1} S^T \; .
\end{align*}




\end{enumerate}


\end{document}



