\subsection{Part 4}
\begin{frame}
    \frametitle{Outline}
    \tableofcontents[currentsection]
\end{frame}

\begin{frame}
    \frametitle{Stability theorem proof, part 4}

    We want to prove Goodwin's technical lemma, which states that $\| \phi(k) \|$ remains bounded and
    \begin{align*}
        \lim_{k \rightarrow \infty} \epsilon(k) = 0
    \end{align*}
    \pause

    This proof will be done in three steps:
    \begin{enumerate}
        \item
        Show that $\epsilon(k)$ remains bounded
        
        \item
        Show that $\| \phi(k) \|$ remains bounded
        
        \item
        Show that $\epsilon(k) \longrightarrow 0$
    \end{enumerate}
\end{frame}

\begin{frame}
    \frametitle{Stability theorem proof, part 4, step 1 ($\epsilon(k)$ bounded)}    

    Recall from part 2 that
    \begin{align*}
        \lim_{k \rightarrow \infty} \frac{ [\lambda_1(k-1) \epsilon(k)]^2 }
            { \lambda_1(k-1) + \phi^T(k-\drm) F(k-1) \phi(k-\drm) } = 0
    \end{align*}
    \pause
    
    Since $0 < \underline{\lambda}_1 \leq \lambda_1(k) \leq 1$ \\
    and $0 < \lambda_{min}(F(k-1)) \leq \lambda_{max}(F(k-1)) \leq K_{max}$ \\
    we have
    \begin{multline*}
        \left| \frac{ [\lambda_1(k-1) \epsilon(k)]^2 }
            { \lambda_1(k-1) + \phi^T(k-\drm) F(k-1) \phi(k-\drm) } \right| \\
        \geq \frac{ \underline{\lambda}_1^2 \epsilon^2(k) }{ 1 + K_{max} \| \phi(k-\drm) \|^2 } > 0
    \end{multline*}
    
\end{frame}

\begin{frame}
    \frametitle{Stability theorem proof, part 4, step 1 ($\epsilon(k)$ bounded)}
    
    \begin{multline*}
        \left| \frac{ [\lambda_1(k-1) \epsilon(k)]^2 }
            { \lambda_1(k-1) + \phi^T(k-\drm) F(k-1) \phi(k-\drm) } \right| \\
        \geq \frac{ \underline{\lambda}_1^2 \epsilon^2(k) }{ 1 + K_{max} \| \phi(k-\drm) \|^2 } > 0
    \end{multline*}
    \hrule{\hfill}

    For convenience, we define $\displaystyle \overline{\epsilon}(k) \ \max_{j\leq k} |\epsilon(j)|$
    \pause
    
    $\, $
    
    From part 3, we have that $\| \phi(k-\drm) \|^2 \leq [C_1 + C_2 \overline{\epsilon}(k)]^2$, which implies that
    \begin{multline*}
        \left| \frac{ [\lambda_1(k-1) \epsilon(k)]^2 }
            { \lambda_1(k-1) + \phi^T(k-\drm) F(k-1) \phi(k-\drm) } \right| \\
        \geq \frac{ \underline{\lambda}_1^2 \epsilon^2(k) }{ 1 + K_{max} [C_1 + C_2 \overline{\epsilon}(k)]^2 } > 0
    \end{multline*}    

    
\end{frame} 

\begin{frame}
    \frametitle{Stability theorem proof, part 4, step 1 ($\epsilon(k)$ bounded)}

    \begin{multline*}
        \left| \frac{ [\lambda_1(k-1) \epsilon(k)]^2 }
            { \lambda_1(k-1) + \phi^T(k-\drm) F(k-1) \phi(k-\drm) } \right| \\
        \geq \frac{ \underline{\lambda}_1^2 \epsilon^2(k) }{ 1 + K_{max} [C_1 + C_2 \overline{\epsilon}(k)]^2 } > 0
    \end{multline*}
    \hrule{\hfill}
        
    Since 
    \begin{align*}
        \lim_{k \rightarrow \infty} \frac{ [\lambda_1(k-1) \epsilon(k)]^2 }
            { \lambda_1(k-1) + \phi^T(k-\drm) F(k-1) \phi(k-\drm) } = 0
    \end{align*}
    we have
    \begin{align*}
        \lim_{k \rightarrow \infty} \frac{ \underline{\lambda}_1^2 \epsilon^2(k) }
            { 1 + K_{max} [C_1 + C_2 \overline{\epsilon}(k)]^2 } = 0
    \end{align*}

\end{frame} 

\begin{frame}
    \frametitle{Stability theorem proof, part 4, step 1 ($\epsilon(k)$ bounded)}

    \begin{align*}
        \lim_{k \rightarrow \infty} \frac{ \underline{\lambda}_1^2 \epsilon^2(k) }
            { 1 + K_{max} [C_1 + C_2 \overline{\epsilon}(k)]^2 } = 0
    \end{align*}
    \hrule{\hfill}

    Whenever $\displaystyle |\epsilon(k)| = \overline{\epsilon}(k) \geq 1$, we have
    \begin{align*}
        0 & < \frac{ 1 + K_{max} [C_1 + C_2 \overline{\epsilon}(k)]^2 }{ \underline{\lambda}_1^2 \overline{\epsilon}^2(k) } \\
        & = \frac{ 1 + K_{max} C_1^2 }{ \underline{\lambda}_1^2 \overline{\epsilon}^2(k) }
            + \frac{ 2 K_{max} C_1 C_2 }{ \underline{\lambda}_1^2 \overline{\epsilon}(k) } 
            + \frac{ K_{max} C_2^2 }{ \underline{\lambda}_1^2 } \\
        & \leq \frac{1}{\underline{\lambda}_1^2 } [1 + K_{max} C_1^2 + 2 K_{max} C_1 C_2 + K_{max} C_2^2]
    \end{align*}
    \paused
    
    This implies that whenever $\displaystyle |\epsilon(k)| = \overline{\epsilon}(k) \geq 1$, we have
    \begin{align*}
        \frac{ \underline{\lambda}_1^2 \epsilon^2(k) }{ 1 + K_{max} [C_1 + C_2 \overline{\epsilon}(k)]^2 } 
            \geq \frac{ \underline{\lambda}_1^2 }{ 1 + K_{max} [C_1 + C_2]^2 } > 0
    \end{align*}

\end{frame} 

\begin{frame}
    \frametitle{Stability theorem proof, part 4, step 1 ($\epsilon(k)$ bounded)}

    Whenever $\displaystyle |\epsilon(k)| = \overline{\epsilon}(k) \geq 1$, we have
    \begin{align*}
        \frac{ \underline{\lambda}_1^2 \epsilon^2(k) }{ 1 + K_{max} [C_1 + C_2 \overline{\epsilon}(k)]^2 }
            \geq \frac{ \underline{\lambda}_1^2 }{ 1 + K_{max} [C_1 + C_2]^2 } > 0
    \end{align*}
    \hrule{\hfill}
    
    Since
    \begin{align*}
        \lim_{k \rightarrow \infty} \frac{ \underline{\lambda}_1^2 \epsilon^2(k) }
            { 1 + K_{max} [C_1 + C_2 \overline{\epsilon}(k)]^2 } = 0
    \end{align*}
    there can only be a finite number of values of $k$ such that $\displaystyle |\epsilon(k)| = \overline{\epsilon}(k) = \max_{j\leq k} |\epsilon(j)| \geq 1$.
    \pause
    
    $\,$
    
    Therefore, 
    \alignbox{
        \epsilon(k) \textrm{ remains bounded}
    }
    
\end{frame}

\begin{frame}
    \frametitle{Stability theorem proof, part 4, step 2 ($\phi(k)$ bounded)}

    Recall from part 3 that
    \begin{align*}
        \| \phi(k-\drm) \| & \leq C_1 + C_2 \max_{j \leq k} |\epsilon(j)|
    \end{align*}
    \pause
    
    Since $\epsilon(k)$ remains bounded, we immediately see that
    \alignbox{
        \phi(k) \textrm{ remains bounded}
    }

\end{frame}

\begin{frame}
    \frametitle{Stability theorem proof, part 4, step 3 ($\epsilon(k) \longrightarrow 0$)}

    Recall from part 2 that
    \begin{align*}
        \lim_{k \rightarrow \infty} \frac{ \epsilon^2(k) }{ \zeta(k) } = 0
    \end{align*}
    where
    \begin{align*}
        \zeta(k) = \frac{\lambda_1(k-1) + \phi^T(k-\drm) F(k-1) \phi(k-\drm)}{\lambda_1^2(k-1)}
    \end{align*}
    \pause
    
    Therefore, if we can show that $\zeta(k)$ remains bounded, it must be true that $\epsilon(k) \longrightarrow 0$

\end{frame}

\begin{frame}
    \frametitle{Stability theorem proof, part 4, step 3 ($\epsilon(k) \longrightarrow 0$)}

    Since $0 < \underline{\lambda}_1 \leq \lambda_1(k) \leq 1$ \\
    and $0 < \lambda_{min}(F(k-1)) \leq \lambda_{max}(F(k-1)) \leq K_{max}$ \\
    we have
    \begin{align*}
        |\zeta(k)| & = \left| \frac{\lambda_1(k-1) + \phi^T(k-\drm) F(k-1) \phi(k-\drm)}{\lambda_1^2(k-1)} \right| \\
        & \leq \frac{ 1 + K_{max} \| \phi(k-\drm) \|^2 }{ \underline{\lambda}_1^2 }
    \end{align*}
    \pause
    
    Since the right-hand side is bounded, we see that $\zeta(k)$ remains bounded.
    \pause
    
    Therefore
    \alignbox{
        \lim_{k \rightarrow \infty} \epsilon(k) = 0
    }

\end{frame}




