\section{Proof, Special Case: $B(q^{-1})$ Anti-Schur}
\begin{frame}
    \frametitle{Outline}
    \tableofcontents[currentsection]
\end{frame}

\begin{frame}
    \frametitle{Proof Methodology}

    The proof will be done in 4 parts:

    \begin{enumerate}
        \item
        Rewrite the system dynamics in a more convenient form
        \pause

        \item
        Prove that $E\{z(k-\drm) \epsilon_f(k)\} = 0$, where $z(k)$ is a sequence to be defined in subsequent slides
        \pause

        \item
        Prove optimality of proposed control scheme
        \pause

        \item
        Verify closed-loop stability

    \end{enumerate}
    \pause

    Comments on the notation in this proof:
    \begin{itemize}
        \item
        Capital letters always denote polynomials; lower case letters denote sequences (except $\drm$ and $q$)

        \item
        Dependency of polynomials on $q^{-1}$ will be omitted\\
        e.g. $\bar{B}^u$ will refer to $\bar{B}^u(q^{-1})$

        \item
        Dependency of sequences on $k$ will be omitted\\
        e.g. $y$ will refer to $y(k)$

    \end{itemize}
\end{frame}

\begin{frame}
    \frametitle{Part 1: Rewrite Dynamics}

    The plant dynamics are
    \begin{align*}
        Ay = \qmd Bu + C \epsilon
    \end{align*}
    and the Diophantine equation gives
    \begin{align*}
        [RA]y = [C - \qmd S] y
    \end{align*}
    \paused

    Combining these two equations gives
    \begin{align*}
        R [\qmd Bu + C \epsilon] = [C - \qmd S] y
    \end{align*}
    \paused
    \begin{align*}
        \Rightarrow \quad Cy - \qmd(Sy + BRu) - CR\epsilon = 0
    \end{align*}

\end{frame}

\begin{frame}
    \frametitle{Part 1: Rewrite Dynamics}

    From the previous slide:
    \begin{align*}
        \quad Cy - \qmd(Sy + BRu) - CR\epsilon = 0
    \end{align*}

    \begin{itemize}
        \item
        Define $z(k)$ in terms of $y(k)$ and $u(k)$ using
        \begin{align*}
            Cz = Sy + BRu
        \end{align*}
        (note that we are not necessarily using the optimal control)
        \pause

        \item
        Define $\epsilon_f = R\epsilon$
        \pause

        \item
        From the top equation,
        \begin{gather*}
            Cy - \qmd Cz - C\epsilon_f = 0 \qquad \Rightarrow \qquad C (y - \qmd z - \epsilon_f) = 0
        \end{gather*}
        \pause
        Since $C$ is anti-Schur, we have $y - \qmd z - \epsilon_f \longrightarrow 0$
        \pause

        \alignbox{
            y(k) = z(k-\drm) + \epsilon_f(k)
        }
    \end{itemize}
\end{frame}

\begin{frame}
    \frametitle{Part 2: $E\{z(k-\drm)\epsilon_f(k)\} = 0$}

    \begin{itemize}
        \item
        Since $\epsilon(k) = y(k) - E\{y(k) | y(k-1),y(k-2),\ldots\}$, we use least squares property 1 to see that
        \begin{align*}
            E \{ y(k-\ell) \epsilon(k+p) \}, \qquad \forall \ell > 0, p \geq 0
        \end{align*}
        \paused

        \item
        $\epsilon_f(k+\drm-1) = \epsilon(k+\drm-1) + r_1 \epsilon(k+\drm-2) + \cdots + r_{\drm-1} \epsilon(k)$
        \begin{align*}
            \Rightarrow \quad E \{ y(k-\ell) \epsilon_f(k+\drm-1) \} = 0 \qquad \forall \ell > 0
        \end{align*}
        \paused

        \item
        Since $u(k)$ is a function of $y(k),y(k-1),\ldots$
        \begin{align*}
            E \{ u(k-\ell) \epsilon_f(k+\drm-1) \} = 0 \qquad \forall \ell > 0
        \end{align*}
        \paused

        \item
        Since $z(k)$ is a function of $y(k),y(k-1),\ldots$ and $u(k),u(k-1),\ldots$
        \begin{align*}
            E \{ z(k-\ell) \epsilon_f(k+\drm-1) \} = 0 \qquad \forall \ell > 0
        \end{align*}
        \paused

        \item
        Choosing $\ell = 1$ completes part 2
    \end{itemize}
\end{frame}

\begin{frame}
    \frametitle{Part 3: Optimal Control}

    Recall that
    \begin{gather*}
        y(k) = z(k-\drm) + \epsilon_f(k) \\
        E\{z(k-\drm)\epsilon_f(k)\} = 0
    \end{gather*}
    \paused
    Therefore,
    \begin{align*}
        E\{y^2(k)\} = E\{z^2(k-\drm)\} + E\{\epsilon_f^2(k)\}
    \end{align*}
    \paused

    Notes:
    \begin{itemize}
        \item
        At this point, we have only assumed that $u(k)$ stabilizes the system (so that the relevant covariances are bounded)
        \pause

        \item
        $E\{\epsilon_f^2(k)\}$ does not depend on the choice of the control law
        \pause

        \item
        $\Rightarrow$ We would like to choose $u$ to minimize $E\{z^2\}$
        \pause

        \item
        If we can make $E\{z^2\} = 0$, the control must be optimal
    \end{itemize}

\end{frame}

\begin{frame}
    \frametitle{Part 3: Optimal Control}

    Goal: choose $u$ so that $E\{z^2\} = 0$
    \pause

    \begin{itemize}
        \item
        If we apply the control signal $u_*(k)$ defined by
        \begin{align*}
            BRu_* = - Sy
        \end{align*}
        \paused
        we have
        \begin{align*}
            Cz = BRu_* + Sy = 0
        \end{align*}
        \paused

        \item
        $C(q^{-1})$ is an anti-Schur polynomial $\Rightarrow \ z(k) \longrightarrow 0$.
        \pause

        \item
        For this control signal, $E\{z^2\} = 0$, which means that $u_*(k)$ is optimal, provided that the closed-loop system is stable
        \pause

        \item
        Also note that $E\{y^2\} = E\{\epsilon_f^2\}$, provided that the closed-loop system is stable
    \end{itemize}
\end{frame}

\begin{frame}
    \frametitle{Part 4: Closed-loop stability}

    From the plant dynamics and feedback law, we have
    \begin{align*}
        \begin{bmatrix}
                A & -\qmd B \\
                S & BR
            \end{bmatrix} \begin{bmatrix}
                y \\
                u
            \end{bmatrix} = \begin{bmatrix}
                C\epsilon \\
                0
            \end{bmatrix}
    \end{align*}
    \pause
    \begin{align*}
        \Rightarrow \begin{bmatrix}
                y \\
                u
            \end{bmatrix} & = \frac{1}{BAR + \qmd BS} \begin{bmatrix}
                BR & \qmd B \\
                -S & A
            \end{bmatrix} \begin{bmatrix}
                C\epsilon \\
                0
            \end{bmatrix} \\
        & = \frac{1}{B(AR + \qmd S)} \begin{bmatrix}
                BRC\epsilon \\
                -SC\epsilon
            \end{bmatrix} = \frac{1}{BC} \begin{bmatrix}
                BRC\epsilon \\
                -SC\epsilon
            \end{bmatrix}
    \end{align*}
    \paused

    Since $C(q^{-1}) B(q^{-1})$ is an anti-Schur polynomial of $q^{-1}$, the closed-loop system is stable
    \hfill $\blacksquare$

\end{frame}

